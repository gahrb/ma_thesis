\chapter{Conclusion and Outlook}

\section{Conclusion}
\label{sec:conclusion}
This thesis showed that the use of passive relays with an optimal choice of relay and matching network impedances can improve the achievable rates significantly, so that current interference eliminating protocols are outperformed.

The results were compared to receiver structures with widely spaced antennas and optimal choices of the matching network with the same number of antennas as ...
Rates close to these limits were achieved.


\section{Future Work}
\label{sec:outlook}
Because this thesis looked into the achievable improvements by the use of passive relays and showed that this method is an auspicious method in diminishing interference for dense networks, the feasibility remains to be prooven.

That the method will be implemented one day, the results must be scaled to larger systems.
Further the time to optimize  the network for one setting must be decreased dramatically.
Therefore different solver approaches must be found and analyzed.

An analysis of the optimal impedance values would most certainly lead to a better understanding of the whole problem.
Maybe correlations to the antenna placing can be found.

The choice of complex components at the relays (i.e. lossy relays), must be further explored, as it gives a higher degree of freedom for the solution finding.

