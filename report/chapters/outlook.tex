\chapter{Conclusion and Outlook}

\section{Conclusion}
\label{sec:conclusion}
This thesis shows that the use of passive relays with an optimal choice of relay and matching network impedances can improve the achievable rates significantly.
It was not only shown that current interference eliminating protocols are outperformed by our method, but also that any combination with methods like TDMA leads to far better results than those without using passive relays.

For the optimization of relay loads and the matching network of receivers, different solver approaches have been introduced with promising results.
For the method of gradient search, we derived the partial derivatives of the achievable user rates and the achievable sum rate for any noise contribution, any spatial channel realization, and  different antenna topologies.
The optimization methods we introduced were further improved by the use of an adaptive step size, conjugate gradients method, and heuristic solvers.
Their performance was further optimized by better choices of initial values, post-refinement of the solution, and stepwise optimization. 

We compared the results of the optimization routine to receiver structures with widely spaced antennas and optimal choices of the matching network, assuming the relays to be fully cooperative receiving antennas.
We repeated this for  different number of passive relays, receiving antennas per user, users, relay placings, and input powers.
With just one receiver antenna, we achieved rates close to ideal limits.
We gave a lower bound of the required number of relays to achieve low interference links for all users in a network.
This lower bound was then compared to a four-user system and evaluated.
To achieve this lower bound, the noise-free rates of the systems were calculated and analyzed.

Finnaly, it was shown that, by keeping the sum of relays and receivers fixed we can find similarly good results independent of whether more receivers and less relays or less receivers and more relays were used.
Hence, increasing the number of relays leads to almost the same improvements as increasing the number of receivers per user.

\section{Future Work}
\label{sec:outlook}
This thesis looked into the improvements achievable through the use of passive relays and showed that this method is an auspicious method for diminishing interference in dense networks, but practical feasibility remains to be proven.
For that, the system must be scaled to larger numbers of users and relays, and the speed of the solver must be improved to find satisfyingly good results.
Therefore, different solver approaches could be the solution.
As the coupling is dependent on the distances among antennas, means of determining each antenna position must be found.

Further, an analysis of the optimal impedance values would most certainly lead to a better understanding.
Investigating good choices of initial values would also potentiallly help, so that a normal gradient search routine with less initializations could be used.
Correlations between impedance values and antenna placments could be found, hence again a better optimization routine with educated initial guesses could be found.

The choice of complex components at the relays (i.e. lossy relays), must be further explored, as it gives a higher degree of freedom for finding solutions.
However, this also requires better performance from the optimization routine, in both speed and precision.

Finally, a real system setup for validation of the solver and the system would give certainty that the approach of eliminating interference by the use of passive relays is reasonable.

