\chapter*{Abstract}
\addcontentsline{toc}{chapter}{Abstract}
%\chapter*{Zusammenfassung}
%\addcontentsline{toc}{chapter}{Zusammenfassung}

In current wireless networks there are two limiting factors for transmission rates: fading and interference.
While multiple-input-multiple-output (MIMO) systems reduce the impact of fading, the problem of interference has not yet been satisfyingly solved.
The most common methods for addressing the problem of interference are protocols based on schemes like time-division-~(TD-), frequency-division-~(FD-), or code-division multiple access (CDMA).
The simultaneous advantage and downfall of those schemes is their unique allocation of specific time and/or frequency slots for each user, hence blocking all other users.

In this thesis the problem of interference is addressed by using closely spaced passive relays: antennas with only passive, lossless impedances attached, which interact with the receiving antennas only through coupling.
These relays require no input power to amplify signals.
By using passive elements, the effects of coupling can be changed, and by using multiple passive relays couplings can be used to increase the signal and reduce the interference power to the receivers.

In the following, a description of the system will be derived that allows for an easy and elegant way of analyzing the effects of coupling.
Different solver methods are analyzed and discussed in order to find the optimal choice of relay loads.
As some of the solvers are based on the method of gradient search, the analytical gradient will be derived.
In order to derive the analytical gradient, full channel knowledge of the spatial channel and couplings between receiving elements are assumed.
For the gradient search method, a variety of initial value choices will be compared.
Further, the results are evaluated and compared to different solver methods and previously known interference-limiting methods.
