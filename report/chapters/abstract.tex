\chapter*{Abstract}
\addcontentsline{toc}{chapter}{Abstract}
%\chapter*{Zusammenfassung}
%\addcontentsline{toc}{chapter}{Zusammenfassung}

In nowadays wireless networks there are mainly two factors witch limit the achievable transmission rates - fading and interference.
When multiple-input-multiple-output (MIMO) systems mostly reduce the impact of fading, the problem of interference has not yet been solved satisfyingly.
The most common methods to address the problem of interference are protocols based on schemes like time-division-~(TD-), frequency-division-~(FD-), or code-division multiple access (CDMA).
The solution - but in the same time also the downside - of those schemes is the unique allocation of a specific time and/or frequency slot for a single user, hence the blocking of all other users.
\paragraph{}
In this thesis the problem of interference is addressed by the use of closely spaced passive relays, i.e. antennas with only passive, lossless impedances attached, which interact with the receiving antennas only by the effect of coupling.
Therefore they require no input power to amplify the signal.
By the choice of the passive elements the effect of the coupling can be changed and hence by using multiple passive relays, the coupling can be used to increase the signal- and reduce the interference power at the receivers.
\paragraph{}
In the following, a description of the system will be derived, which allows an easy and elegant way of analyzing the effect of coupling.
Different solver methods are analyzed and discussed in order to find an optimal choice of the relay impedances.
As some of the solvers are based on the method of gradient search, the analytical gradient will be derived.
For the gradient search method, a variety of initial value choices will be compared to each other.
Further, the results are evaluated and compared to different solver methods and previously known interference limiting methods.
